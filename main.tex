%-----------------------------------------------------------------
%	BASIC DOCUMENT LAYOUT
%-----------------------------------------------------------------
\documentclass[paper=a4, fontsize=12pt, twoside=semi]{scrartcl}
\usepackage[T1]{fontenc}
\usepackage[utf8]{inputenc}
\usepackage{lmodern}
\usepackage{slantsc}
\usepackage{microtype}
\usepackage[british]{babel}

% Bibliography
\usepackage[backend=bibtex, style=trad-abbrv, maxbibnames=5]{biblatex}
\addbibresource{bibliografia.bib}

% Sectioning layout
\addtokomafont{sectioning}{\normalfont\scshape}
\usepackage{tocstyle}
\usetocstyle{standard}
\renewcommand*\descriptionlabel[1]{\hspace\labelsep\normalfont\bfseries{#1}}

% Empty pages
\usepackage{etoolbox}
% \pretocmd{\section}{\cleardoubleevenemptypage}{}{}
\pretocmd{\part}{\cleardoubleevenemptypage\thispagestyle{empty}}{}{}
\renewcommand\partheadstartvskip{\clearpage\null\vfil}
\renewcommand\partheadmidvskip{\par\nobreak\vskip 20pt\thispagestyle{empty}}

% Paragraph indentation behaviour
\setlength{\parindent}{0pt}
\setlength{\parskip}{0.3\baselineskip plus2pt minus2pt}
\newcommand{\sk}{\medskip\noindent}

% Fancy header and footer
\usepackage{fancyhdr}
\pagestyle{fancyplain}
\fancyhead[LO]{\thepage}
\fancyhead[CO]{}
\fancyhead[RO]{\nouppercase{\mytitle}}
\fancyhead[LE]{\nouppercase{\leftmark}}
\fancyhead[CE]{}
\fancyhead[RE]{\thepage}
\fancyfoot{}
\fancypagestyle{abstractpage}
{
	\renewcommand{\headrulewidth}{0pt}
	\fancyhf{}
	% \fancyfoot[C]{\tiny{Random phrase}}
}
\renewcommand{\headrulewidth}{0.3pt}
\renewcommand{\footrulewidth}{0pt}
\setlength{\headheight}{13.6pt}

%-----------------------------------------------------------------
%	MATHS AND SCIENCE
%-----------------------------------------------------------------
\usepackage{amsmath,amsfonts,amsthm,amssymb}
\usepackage{xfrac}
\usepackage[a]{esvect}
\usepackage{chemformula}
\usepackage{graphicx}

\usepackage[arrowdel]{physics}
	\renewcommand{\vnabla}{\vec{\nabla}}
	% \renewcommand{\vectorbold}[1]{\boldsymbol{#1}}
	% \renewcommand{\vectorarrow}[1]{\vec{\boldsymbol{#1}}}
	% \renewcommand{\vectorunit}[1]{\hat{\boldsymbol{#1}}}
	\renewcommand{\vectorarrow}[1]{\vec{#1}}
	\renewcommand{\vectorunit}[1]{\hat{#1}}
	\renewcommand*{\grad}[1]{\vnabla #1}
	\renewcommand*{\div}[1]{\vnabla \vdot \va{#1}}
	\renewcommand*{\curl}[1]{\vnabla \cp \va{#1}}
	\let\rot\curl

% SI units
\usepackage[separate-uncertainty=true]{siunitx}
\sisetup{range-phrase = \text{--}, range-units = brackets}
\DeclareSIPrePower\quartic{4}
	%\DeclareSIUnit\micron{\micro\metre}

% Smaller trig functions
\newcommand{\Sin}{\trigbraces{\operatorname{s}}}
\newcommand{\Cos}{\trigbraces{\operatorname{c}}}
\newcommand{\Tan}{\trigbraces{\operatorname{t}}}

% Operator-style notation for matrices
\newcommand*{\mat}[1]{\hat{#1}}

% Matrices in (A|B) form via [c|c] option
\makeatletter
\renewcommand*\env@matrix[1][*\c@MaxMatrixCols c]{%
	\hskip -\arraycolsep
	\let\@ifnextchar\new@ifnextchar
	\array{#1}}
\makeatother

% Shorter \mathcal and \mathbb
\newcommand*{\mc}[1]{\mathcal{#1}}
\newcommand*{\mbb}[1]{\mathbb{#1}}

% Shorter ^\ast and ^\dagger
\newcommand*{\sast}{^{\star}{}}
\newcommand*{\sdag}{^{\dagger}{}}

% Complex and Hermitian conjugates
\newcommand*{\cc}{\,\text{c.c.}}
\newcommand*{\Hc}{\,\text{H.c.}}

% Blackboard bold identity
\usepackage{bbm}
\newcommand*{\bbid}{\mathbbm{1}}

% Shorter displaystyle
\newcommand*{\dsp}{\displaystyle}

%-----------------------------------------------------------------
%	OTHER PACKAGES
%-----------------------------------------------------------------
\usepackage{environ}

% Plots and graphics
\usepackage{pgfplots}
\usepackage{tikz}
	\usetikzlibrary{calc}
\usepackage{color}
	\makeatletter
		\color{black}
		\let\default@color\current@color
	\makeatother

% Richer enumerate, figure, and table support
\usepackage{enumerate}
\usepackage[shortlabels]{enumitem}
\usepackage{float}
\usepackage{tabularx}
\usepackage{booktabs}
	%\setlength{\intextsep}{8pt}
% \numberwithin{equation}{section}
% \numberwithin{figure}{section}
% \numberwithin{table}{section}

% No indentation after certain environments
\makeatletter
\newcommand*\NoIndentAfterEnv[1]{%
	\AfterEndEnvironment{#1}{\par\@afterindentfalse\@afterheading}}
\makeatother
% \NoIndentAfterEnv{thm}
\NoIndentAfterEnv{defi}
\NoIndentAfterEnv{example}
\NoIndentAfterEnv{table}

% Misc packages
\usepackage{lipsum}

%-----------------------------------------------------------------
%	THEOREMS
%-----------------------------------------------------------------
\usepackage{thmtools}

% Theroems layout
\declaretheoremstyle[
	spaceabove=6pt, spacebelow=6pt,
	headfont=\normalfont,
	notefont=\mdseries, notebraces={(}{)},
	bodyfont=\small,
	postheadspace=1em,
]{small}

\declaretheorem[style=plain,name=Teorema,qed=$\square$,numberwithin=section]{thm}
\declaretheorem[style=plain,name=Corol·lari,qed=$\square$,sibling=thm]{cor}
\declaretheorem[style=plain,name=Lema,qed=$\square$,sibling=thm]{lem}
\declaretheorem[style=definition,name=Definició,qed=$\blacksquare$,numberwithin=section]{defi}
\declaretheorem[style=definition,name=Exemple,qed=$\blacktriangle$,numberwithin=section]{example}
\declaretheorem[style=small,name=Demostració,numbered=no,qed=$\square$]{sproof}

%-----------------------------------------------------------------
%	ELA MOTHERFUCKING GEMINADA
%-----------------------------------------------------------------
\def\xgem{%
	\ifmmode
		\csname normal@char\string"\endcsname l%
	\else
		\leftllkern=0pt\rightllkern=0pt\raiselldim=0pt
		\setbox0\hbox{l}\setbox1\hbox{l\/}\setbox2\hbox{.}%
		\advance\raiselldim by \the\fontdimen5\the\font
		\advance\raiselldim by -\ht2
		\leftllkern=-.25\wd0%
		\advance\leftllkern by \wd1
		\advance\leftllkern by -\wd0
		\rightllkern=-.25\wd0%
		\advance\rightllkern by -\wd1
		\advance\rightllkern by \wd0
		\allowhyphens\discretionary{-}{}%
		{\kern\leftllkern\raise\raiselldim\hbox{.}%
			\kern\rightllkern}\allowhyphens
	\fi
}
\def\Xgem{%
	\ifmmode
		\csname normal@char\string"\endcsname L%
	\else
		\leftllkern=0pt\rightllkern=0pt\raiselldim=0pt
		\setbox0\hbox{L}\setbox1\hbox{L\/}\setbox2\hbox{.}%
		\advance\raiselldim by .5\ht0
		\advance\raiselldim by -.5\ht2
		\leftllkern=-.125\wd0%
		\advance\leftllkern by \wd1
		\advance\leftllkern by -\wd0
		\rightllkern=-\wd0%
		\divide\rightllkern by 6
		\advance\rightllkern by -\wd1
		\advance\rightllkern by \wd0
		\allowhyphens\discretionary{-}{}%
		{\kern\leftllkern\raise\raiselldim\hbox{.}%
			\kern\rightllkern}\allowhyphens
	\fi
}

\expandafter\let\expandafter\saveperiodcentered
	\csname T1\string\textperiodcentered \endcsname

\DeclareTextCommand{\textperiodcentered}{T1}[1]{%
	\ifnum\spacefactor=998
		\Xgem
	\else
		\xgem
	\fi#1}

%-----------------------------------------------------------------
%	PDF INFO AND HYPERREF
%-----------------------------------------------------------------
\usepackage[hypertexnames=false]{hyperref}
\hypersetup{colorlinks, citecolor=black, filecolor=black, linkcolor=black, urlcolor=black}
\usepackage{cleveref}
	\crefname{section}{\S}{\SS}
	\Crefname{section}{\S}{\SS}

\newcommand*{\mytitle}{Using Genetic Algorithms for Forest Fire Spread Prediction: }
\newcommand*{\mysubtitle}{Analyzing the impact of the fitness function}
\newcommand*{\myauthor}{Cristià Estany Bagué}
\newcommand*{\myuni}{Universitat Autònoma de Barcelona, Departament d'Arquitectura de Computadors i Sistemes Operatius}
\newcommand*{\mydate}{\normalsize \today}
\newcommand*{\myadvisor}{\small Advisor: Anna Cortés }

\pdfstringdefDisableCommands{\def\and{i }}

\usepackage{hyperxmp}
\hypersetup{pdfauthor={\myauthor}, pdftitle={\mytitle}}

%-----------------------------------------------------------------
%	TITLE SECTION AND DOCUMENT BEGINNING
%-----------------------------------------------------------------
\newcommand{\horrule}[1]{\rule{\linewidth}{#1}}
\title{
	\normalfont
	\small \scshape{\myuni} \\ [25pt]
	\horrule{0.5pt} \\[0.4cm]
	\huge \mytitle \\
	\Large \scshape{\mysubtitle} \\
	\horrule{2pt} \\[0.5cm]
}
\author{\myauthor}
\author{\myauthor\\[1cm]{\myadvisor}}
\date{\mydate}

\begin{document}

\clearpage\maketitle
\thispagestyle{empty}
\addtocounter{page}{-1}

%-----------------------------------------------------------------
%	DOCUMENT BODY
%-----------------------------------------------------------------
\cleardoubleevenemptypage
\thispagestyle{abstractpage}
%-----------------------------------------------------------------
%	ABSTRACT
%	!TEX root = ./../main.tex
%-----------------------------------------------------------------
\begin{abstract}
	\lipsum[1]
\end{abstract}


% \cleardoubleevenemptypage
% \pdfbookmark[1]{\contentsname}{toc}
% \tableofcontents

\cleardoubleevenemptypage
%-----------------------------------------------------------------
%	!TEX root = ./../main.tex
%-----------------------------------------------------------------
\section{Introduction}\label{sec:intro}

Natural hazards cause annually immense losses in structural damage to property and to person, even taking lives in the process. Particularly, forest fires are among the most troubling because of their increasing number of occurrences. In order to prevent these damages, over the last decades, several physical and mathematical models have been developed and implemented by research groups from a wide range of disciplines to simulate the behaviour of these hazards. Most of these models are very sensitive to the provided input data, unfortunately in most cases this data is missing, outdated and difficult to obtain. Because of this, usually incomplete data have to be supplemented with interpolated data or data predicted by extra models leading to unavoidable uncertainties. These uncertainties have to be minimized because they can have a huge impact in the quality of the prediction.

In our case, data can be classified in two groups depending on its variability with respect to time: static data and dynamic data. Static data corresponds to those parameters that remain reasonably constant during the simulation, such as most of the data related to topology. On the other hand, dynamic data are those parameters that significantly vary during the simulation such as data related to weather, wind and humidity. Both types of data are equally important for the predictive power of the model but, in our case, correctly adjusting dynamic data will be paramount and we will see it clearly with an example. While outdated or interpolated data about the profile of the terrain could be good enough because this does not vary much with years, on the other hand, outdated data about weather or wind in most cases will be completely useless because it can change dramatically within hours. In order to minimize the uncertainty in this case, a Dynamic Data Driven System called Two-Stage Prediction has been developed.

This method proposes to use a Genetic Algorithm during an early stage of the simulation called "calibration step". In this Genetic Algorithm the individuals are different simulations of the same forest fire and the genes are the parameters that we want to improve. The fittest individual is then used for the rest of the simulation. However, this method has two points that have to be taken into account. First, we need, not only the initial and the final perimeter of the fire, but an extra perimeter corresponding to the calibration step. And second, like all the Genetic Algorithms, the result is dependent on the fitness or error function used, so we will need to choose one that is appropriate for our purposes. What is appropriate for our case is something we will discuss in the corresponding section, but, just to give a bit of insight, we would like our simulation to overestimate the burnt area, rather than underestimate, because the purpose of this model is to ease the task of the fire fighters. To this aim, we cannot underestimate the burnt area, because then the fire could persist significantly increasing the damages we mentioned earlier.

The aim of this paper will be to find the best error function that maximizes the quality of the prediction. For this purpose, we will implement several error formulas for each of the forest fires we have data of, and see whether there exists a formula that outperforms the others in all the cases or if it depends on the study case.

%-----------------------------------------------------------------


%-----------------------------------------------------------------
%	BIBLIOGRAPHY
%-----------------------------------------------------------------
% \nocite{b:}

\printbibliography

% \setcounter{secnumdepth}{0}
% \section{Bibliografia}
% \printbibliography[title={En línia}, keyword=web, heading=subbibliography]
% \printbibliography[title={Bàsica}, keyword=basic, heading=subbibliography]
% \printbibliography[title={Avançada}, keyword=advanced, heading=subbibliography]

%-----------------------------------------------------------------
\end{document}
